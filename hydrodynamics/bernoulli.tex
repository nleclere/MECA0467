\section{Governing equations}
\label{sec:hydraulics:governing}

We first specialize the governing equations to constant-density flows,
such as encountered in hydraulics or very low-speed gas flows. The
Navier-Stokes equations in the relative frame reduce to:
\begin{align*}
  \dive \aVelV &= 0 \\
  \ddt{\aVelV} 
  + \dive \aVelV \aVelV 
  + \frac{\grad \pres}{\dens}  &= 
  \frac{\dive \shearStressT}{\dens} \\
  \ddt{\tEner} 
  + \dive \aVelV \tEnth 
  &= \aVelV \cdot \frac{\dive \shearStressT}{\dens} 
  + \frac{\dive \heatFluxV}{\dens}
\end{align*}

\subsection{Bernoulli's equation}

Starting from \ref{eq:bernoulli2}, we find that Bernoulli's equation
for incompressible flow
\begin{align*}
  d \left(\frac{\vel^2}{2} + \grav z + \frac{\pres}{\dens} \right) &=
  d\xyzV \cdot \frac{\dive \shearStressT}{\dens}
\end{align*}
indicates that in the absence of viscous work, the sum of mechanical
energy and (reduced) pressure is conserved along a streamline. This
sum is defined up to a constant, and therefore, we can choose a
convenient reference pressure $\pres^\ast$ and height $z^\ast$ to
define the hydraulic energy
\begin{align*}
  \eHydr = \frac{\pres - \pres^\ast}{\dens} + \frac{\vel^2}{2} + \grav
  \left(z - z^\ast\right)
\end{align*}
For hydraulic engineering, gravity potential differences are important
with respect to pressure and dynamic effects. In this case we use the
\emph{total head} or equivalent \emph{liquid column height}
\begin{align}
  \head = \frac{\pres - \pres^\ast}{\dens \grav} + \frac{\vel^2}{2
    \grav} + \left(z - z^\ast\right)
  \label{eq:hydraulicEnergy}
\end{align}
%We refer to the section \ref{sec:hydraulicCircuits} on hydraulics.
Within the turbomachine, due to the important dynamic effects and
relatively small distances, we can neglect differences in gravity
potential. In this case typically the \emph{total} or \emph{stagnation
  pressure} is used, defined as:
\begin{align*}
  \sPres{} = \pres - \pres^\ast + \frac{1}{2} \dens \vel^2
\end{align*}

\subsection{Irreversibility} 

Due to incompressibility the conservation equation for kinetic energy
\ref{eq:conservationKineticEnergy} can be manipulated to a
conservation equation for the specific total hydraulic energy:
\begin{equation}
  \ddt{\eKin}  + 
  \dive \aVelV \eHydr =
  \dive \shearStressT \cdot \aVelV - \shearStressT : \grad \aVelV 
  \label{eq:conservationHydraulicEnergy}
\end{equation}
which indicates that in the absence of friction the flux of hydraulic
energy is conserved. Indeed, assuming a steady volumetric flow
\vFlow~in a device, we find the following balances for the average
hydraulic energy between inlet (1) and outlet (2) by integrating the
above equations on the volume $\vol$:
\begin{align*}
  \vFlow \left(\left(\overline{\eHydr}\right)_2 - \left(\overline{\eHydr}\right)_1\right) &
  = - \int_\vol \shearStressT : \grad \aVelV~dV 
\end{align*}
Hereby we assume that the viscous stress performs no (notable) work on
in- and outlet and no heat was transferred across the walls.  The
equation \ref{eq:conservationInternalEnergy} for the internal energy
\ener~simplifies to
\begin{equation}
  \ddt{\ener} + \dive \aVelV \ener = 
  \shearStressT : \grad \aVelV + \dive \heatFluxV 
\end{equation}\\
we find the following balance for the internal energy
\begin{align*}
  \vFlow \left(\overline{\ener_2} - \overline{\ener_1}\right) &
  = \int_\vol \shearStressT : \grad \aVelV~dV 
\end{align*}
We see that the average loss of hydraulic energy directly translates
in the increase of thermal energy. In fact, losses of total hydraulic
energy can directly be measured by the temperature increase between
in- and outlet:
\begin{align*}
  \heatCapV \Delta \overline{\temp} = - \Delta \overline{\eHydr}
\end{align*}

\subsection{Enthalpy and entropy}

The enthalpy for an incompressible fluid is defined up to a constant:
\begin{align*}
  \enth = \heatCapV \left(\temp - \temp^\ast\right) + \frac{\pres - \pres^\ast}{\dens} 
\end{align*}
The entropy is only determined by temperature
\begin{align*}
  \temp d \entr = d \ener &\Rightarrow 
  \entr - \entr^\ast = \heatCapV \ln \left(\frac{T}{T^\ast}\right) \\
  &\Rightarrow T = T^\ast e^{\frac{\entr-\entr^\ast}{\heatCapV}}
\end{align*} 
while the pressure is given by an exponential relationship between
$\enth$ and $\entr$:
\begin{align*}
  \left(\enth\right)_{p} = \frac{\pres - \pres^\ast}{\dens} + T^\ast e^{\frac{\entr-\entr^\ast}{C_v}}
\end{align*}
These two relations can be used to construct the Mollier $\enth-\entr$
diagram, used for evaluating the efficiency of flow processes in
particular within turbomachinery. 

\subsection{Formulation in a moving frame of reference}

All of the above equations can be easily extended to their relative
form by adding the coriolis and centrifugal force in the momentum
equations and generalizing the hydraulic energy to
\begin{align*}
  \eHydr^R = 
  \frac{\rVel^2}{2} - 
  \frac{\fVel^2}{2} + 
  \frac{\pres - \pres^\ast}{\dens} + 
  \grav \left(z-z^\ast\right)
\end{align*}
This is left as an exercise.

%%% Local Variables: 
%%% mode: latex
%%% TeX-master: t
%%% End: 
