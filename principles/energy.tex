\section{Energy distribution and irreversibility}

In the following sections we manipulate the Navier-Stokes equations to
analyse the conversion of different forms of energy into one
another. Thereby we will use the following identity
\begin{equation}
  \begin{split}
    \dens \ddtM{a} = \ddt{ \dens a} + \dive \rho \aVelV a
  \end{split}
  \label{eq:materialToConservative}
\end{equation}
which follows from the definition of the material derivative and the
continuity equation \ref{eq:massConservationDifferential}.

\subsection{Mechanical energy}

In order to find an equation for the mechanical energy, we will
premultiply the momentum equation, which expresses the balance of
forces on a fluid particle, with the velocity. From the inertial and
convection term, we find the material derivative of the specific
kinetic energy $\eKin = \aVel^2/2$:
\begin{align*}
  \aVelV \cdot \left(\ddt{\rho \aVelV} + \rho \aVelV \aVelV \right)
  &= 
  \left(2 \eKin \ddt{\rho} + \rho \ddt{\eKin}  \right) + 
  \left(2 \eKin \dive \rho \aVelV + \rho \aVelV \grad \eKin \right) \\
  &= 
  2 \eKin \underbrace{\left(\ddt{\rho} + \dive \rho \aVelV \right)}_{=0} + 
  \left( \rho \ddt{\eKin} + \rho \aVelV \grad \eKin\right)\\
  &= 
  \rho \ddt{\eKin} + \dens \aVelV \cdot \grad \eKin = \rho \ddtM{\eKin} \\
\end{align*}
The work against gravity can similary be reorganized as a material
derivative for the gravity potential:
\begin{align*}
  \aVelV \cdot \rho \gravV = \dens \grav \ddtM{z}  = \dens \ddtM{\eGrav}
\end{align*}
and therefore we find the following equation for the material
derivative of the specific mechanical energy $\eMech = \eKin + \grav
z$:
\begin{equation}
  \dens \ddtM{\eMech} = - \aVelV \cdot \grad p +\aVelV \cdot \dive \shearStressT
  \label{eq:bernoulli}
\end{equation}
which multiplied with $dt$ gives us the differential equation for displacements along
the stream line:
\begin{equation}
  \begin{split}
    d \eMech &= - 
    d\xyzV \cdot \frac{\grad p}{\dens} + 
    d\xyzV \cdot \frac{\dive \shearStressT}{\dens} \\
    &= dW_{rev} + dW_{irr} 
  \end{split}
  \label{eq:bernoulli2}
\end{equation}
This is the generic form of the Bernoulli equation. It describes the
evolution of the mechanical energy of a parcel of fluid under the
influence of the specific work performed against the resulting force,
exerted on the parcel through stress by the neighbouring
parcels. Indeed, we recall that for any control volume $\vol$:
\begin{align*}
  \int_\vol \grad p dV = \int_{\srf} p \nrmV ~dS = \forceV_p \\
  \int_\vol \dive \shearStressT dV = \int_{\srf} \shearStressT \cdot \nrmV ~dS = \forceV_\shearStress 
\end{align*}
It should be noted that both contributions, including the viscous
term, can be positive or negative depending on the alignment of the
resulting force and the velocity. Indeed, a slow parcel of fluid can
be accelerated by adjacent faster parcels through friction.

\subsection{Thermal energy}

For further manipulation of the energy equation, we first derive an
additional conservation equation for the mechanical energy from
equation \ref{eq:bernoulli}. First we rearrange the work terms:
\begin{align*}
  \aVelV \cdot \grad p &= \dive \pres \aVelV - \pres \dive \aVelV \\
  \aVelV \cdot \dive \shearStressT &= \dive \shearStressT \cdot \aVelV - \shearStressT : \grad \aVelV  
\end{align*}
which then, using identity \ref{eq:materialToConservative} for $a =
\eMech$, gives:
\begin{equation}
  \ddt{\dens \eMech} + 
  \dive \dens \aVelV \eMech + 
  \dive \aVelV p = p \dive \aVelV +
  \dive \shearStressT \cdot \aVelV - \shearStressT : \grad \aVelV 
  \label{eq:conservationKineticEnergy}
\end{equation}
Subtracting this expression from the energy equation
\ref{eq:energyConservationDifferential} we find
\begin{equation}
  \begin{split}
    \ddt{\rho \ener} + \dive \rho \aVelV \ener &= - \pres \dive \aVelV +
    \shearStressT : \grad \aVelV + \dive \heatFluxV \\
    \dens \ddtM{\ener} &= - \pres \dive \aVelV + 
    \shearStressT : \grad \aVelV + \dive \heatFluxV 
  \end{split}
  \label{eq:conservationInternalEnergy}
\end{equation}
Defining the specific volume $\sVol = 1/\dens$, we find
\begin{align*}
  \dive \aVelV &= - \frac{1}{\dens} \left(\ddt{\dens} + \aVelV \cdot \grad \dens\right) = - \frac{1}{\dens} \ddtM{\dens} = \dens \ddtM{\sVol}
\end{align*}
we find
\begin{equation}
  \begin{split}
  \dens \ddtM{\ener} &= - \dens \pres \ddtM{\sVol} +
  \shearStressT : \grad \aVelV + \dive \heatFluxV 
  \end{split}
  \label{eq:materialInternalEnergy}
\end{equation}
which shows that the increase in the static internal energy is equal
to the sum of the compression work, the viscous dissipation and the
heat flux.

Now we can easily show irreversibility. We recall that the velocity
gradient is decomposed in a symmetric and antisymmetric tensor:
\begin{align*}
  \grad \aVelV = \shearRateT + \rotRateT
\end{align*}
The shear stress tensor is then defined as
\begin{align*} 
  \shearStressT = 2 \dynVisc \shearRateT - \bulkVisc \left(\dive \aVelV\right) \identityT \\
\end{align*} 
Since $\shearStressT$ and $\shearRateT$ are symmetric, and
$\rotRateT$ is antisymmetric, we have
\begin{align*}
  \shearStressT : \grad \aVelV &= \shearStressT : \shearRateT = 2
  \dynVisc \shearRateT : \shearRateT - \left(\bulkVisc \dive \aVelV
  \right)^2
\end{align*}
From there on, it is left as an exercise that always $\shearStressT :
\grad \aVelV >= 0$ for the standard value $\bulkVisc = 2/3
\dynVisc$. This means that the viscous stresses always contribute to
the increase of thermal energy, never to its decrease. Hence their
effect is irreversible.

\subsection{Entropy}

The second law of thermodynamics quantifies the irreversibility of
thermodynamic processes. It proceeds by defining the state variable
\emph{entropy} \entr~, whose increase is proportional to the increase
of internal energy $\ener$~minus the (reversible) compression work, or
the sum of the dissipated mechanical energy $dq_{irr}$ and the heat
load $dq$:
\begin{align*}
  \temp d \entr = d\ener + \pres d \sVol = dq_{irr} + dq
\end{align*}
we find from equation \ref{eq:conservationInternalEnergy}
\begin{align*}
  \dens \temp \ddtM{\entr} = \shearStressT : \grad \aVelV + \dive \heatFluxV
\end{align*}
Again using identity \ref{eq:materialToConservative}, we find the
equation for the conservation of entropy:
\begin{equation}
  \ddt{\dens \entr} + \dive \dens \aVelV \entr = 
  \frac{\shearStressT : \grad \aVelV}{\temp} + \frac{\dive \heatFluxV}{\temp}
\end{equation}
which provides us with an expression for $dq_{irr} = \shearStressT :
\grad \aVelV$. We can easily see that in the absence of heat transfer,
entropy can only increase, as expected.

%%% Local Variables: 
%%% mode: latex
%%% TeX-master: t
%%% End: 
