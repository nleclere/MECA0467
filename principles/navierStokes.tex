% Reynolds transport theorem
% mass conservation
% momentum conservation, pressure and shear stress
% energy conservation, heat flux
% thermodynamic EoS - incompressible / compressible / ideal gas
% second law of thermodynamics

\section{The Navier-Stokes equations}

We now apply the Reynolds transport theorem a fluid parcel moving with
velocity $\aVelV$. Therefore we consider its enclosing control volume
$\vol$. Since the control volume moves with the fluid we have
$\aVelV_\vol = \aVelV$. Then we can apply conservation of mass, Newtons
equation of motion (conservation of momentum) and the first law of
thermodynamics (conservation of energy) to the volume, to find the
Navier-Stokes equations.

\subsection{Conservation of mass} 

Since the control volume moves with the fluid, its mass does not
change, and thus
\begin{align*}
  \ddtM{} \int_{\vol} \dens~dV = 0
\end{align*}
Applying the transport theorem we find the \emph{conservative} form of
equation governing the conservation of mass, also called the
\emph{continuity equation}:
\begin{equation}
  \int_{\vol} \ddt{\dens}~dV + \oint_{\srf} \dens (\aVelV \cdot \nrmV)~dS = 0
  \label{eq:massConservationIntegral}
\end{equation}
Using the divergence theorem and the fact that equation
\ref{eq:massConservationIntegral} holds for any parcel of fluid,
however small, we find an equivalent \emph{differential} form:
\begin{equation}
  \ddt{\dens} + \dive \dens \aVelV = 0
  \label{eq:massConservationDifferential}
\end{equation}

\subsection{Conservation of momentum} 

Following Newton's law, the momentum enclosed in the volume changes
through the action of the body forces, including gravity $\gravV$, and
the stress \stressT~on the contour:
\begin{align*}
  \ddtM{} \int_{\vol} \dens \aVelV~dV = \int_{\vol} \dens \gravV~dV +
  \oint_{\srf} (\stressT \cdot \nrmV)~dS
\end{align*}
As discussed before, the stress in the fluid is composed of the
hydrodynamic pressure $p$ and the shear stress $\shearStressT$
\begin{align*}
  \stressT = - \pres \identityT + \shearStressT 
\end{align*}
Applying this decomposition and the transport theorem we find the
conservative form of the momentum equation
\begin{equation}
  \int_{\vol} \ddt {\dens \aVelV}~dV + 
  \oint_{\srf} \left(\dens \aVelV \aVelV + p\right) \cdot \nrmV ~dS = 
  \int_{\vol} \dens \gravV~dV +
  \oint_{\srf} \shearStressT \cdot \nrmV~dS
  \label{eq:momentumConservationIntegral}
\end{equation}
Using the divergence theorem, we then find the differential form 
\begin{equation}
  \ddt {\dens \aVelV} + 
  \dive \dens \aVelV \aVelV + \grad p = \dens \gravV + \dive \shearStressT  
  \label{eq:momentumConservationDifferential}
\end{equation}

\subsection{Conservation of energy}

The specific, \ie per unit of mass, \emph{internal energy} is defined
as
\begin{align*}
  \ener = \heatCapV T
\end{align*}
with \heatCapV~the heat capacity at constant volume and \temp~ the
temperature. The total internal energy \tEner~is then defined as the
sum of the internal energy, the kinetic energy and the gravitational
potential:
\begin{align*}
  \tEner = \ener + \frac{1}{2} \aVel^2 +
  \grav z = \ener + \eKin + \eGrav = \ener + \eMech
\end{align*}
Here we assume $\gravV = - \grav \unitV{z}$, and grouped the kinetic
energy and gravity potential in the mechanical energy $\eMech = \eKin
+ \eGrav$. Obviously, the energy per unit of volume is $\dens \tEner$.

Following the first law of thermodynamics, the change of energy of a
closed volume is equal to the work exerted by the stress $\stressT
\cdot \nrmV$ and the heat flux $\heatFluxV \cdot \nrmV$ through the
surface $\srf$. This law can be applied to the control volume around a
fixed parcel of fluid:
\begin{align*}
  \ddtM{} \int_{\vol} \dens \tEner dV = 
  \oint_{\srf} \aVelV \cdot (\stressT \cdot \nrmV)~dS + 
  \oint_{\srf} \heatFluxV \cdot \nrmV~dS 
\end{align*}
Using the decomposition of the stress together with the transport
equation we find:
\begin{align*}
  \int_{\vol} \ddt{\dens \tEner} dV + 
  \oint_{\srf} \dens \tEner (\aVelV \cdot \nrmV)~dS + 
  \oint_{\srf} p \aVelV \cdot \nrmV~dS = 
  \oint_{\srf} \aVelV \cdot (\shearStressT \cdot \nrmV)~dS +
  \oint_{\srf} \heatFluxV \cdot \nrmV~dS
\end{align*}
Defining the \emph{static} and \emph{total} or \emph{stagnation}
enthalpy resp. as
\begin{align*}
  \enth &= \ener + \frac{\pres}{\dens} \\
  \tEnth &= \tEner + \frac{\pres}{\dens} = \enth + \eKin
\end{align*}
we can rewrite conservative form of the energy equation as
\begin{equation}
  \int_{\vol} \ddt{\dens \tEner} dV + 
  \oint_{\srf} \dens \tEnth (\aVelV \cdot \nrmV)~dS  = 
  \oint_{\srf} \aVelV \cdot (\shearStressT \cdot \nrmV)~dS +
  \oint_{\srf} \heatFluxV \cdot \nrmV~dS
  \label{eq:energyConservationIntegral}
\end{equation}
whereas the differential form describes the time evolution of the
total energy per unit of volume:
\begin{equation}
  \ddt{\dens \tEner} + \dive  \dens \tEnth \aVelV   = 
  \dive \shearStressT \cdot \aVelV + \dive \heatFluxV
  \label{eq:energyConservationDifferential}
\end{equation}


%%% Local Variables: 
%%% mode: latex
%%% TeX-master: t
%%% End: 
