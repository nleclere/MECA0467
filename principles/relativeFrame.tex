
\section{The Navier-Stokes equations in the relative frame}
\label{sec:principles:navierStokesRelative}

The Navier-Stokes equations have been derived in the absolute frame of
reference. We now proceed to formulating these equations with respect
to the rotating frame of reference attached to a turbomachinery rotor. 
\begin{figure}[!h]
  \centering
  \includegraphics[width=0.75\textwidth]{principles/frame.tikz}
  \caption{Relation between velocities in the absolute and the relative frame}
  \label{fig:frameOfReference}
\end{figure}
To this end we will first derive the equations for a generic frame of
reference with speed $\fVelV$. 

\subsection{Transformations from the absolute to the relative frame}

The frame illustrated in figure \ref{fig:frameOfReference} undergoes a
solid body motion, composed of a constant speed $\fVelV_l$ combined to
a rotation $\rotV$ around an (arbitrarily) chosen center $\xyzV_c$:
\begin{align*}
  \fVelV =  
  \fVelV_l + 
  \rotV \times \left(\xyzV - \xyzV_c\right)
\end{align*}

We will now establish the relation between the velocity of a material
point in the absolute frame $\aVelV$ and in the relative frame of
reference $\rVelV$. At time t we align the coordinate systems of both
frames. At $t+dt$, the relative frame has moved over $\fVelV dt +
\mathcal O(dt^2)$, the attached coordinate system is not only
displaced over the same distance but also rotated, to end up at
$(x^\prime,y^\prime)$.

In the absolute frame, the point $P$, initially coinciding with the
material point, and fixed to the relative frame has moved to
$P^\prime$ at $\xyzV^\prime = \xyzV + \fVelV~dt + \mathcal O(dt^2)$;
 the
material point has since moved within the relative frame from $P^\prime$ at $\xyzV^\prime$ to
$\xyzV^{\prime \prime} = \xyzV^\prime + \rVelV~dt + \mathcal
O(dt^2)$, with $\rVelV$ the relative velocity. Therefore the absolute velocity of the material particle is
found from
\begin{align*}
  \aVelV = \lim_{dt\rightarrow 0} \frac{\xyzV^{\prime \prime} - \xyzV}{dt} = \fVelV + \rVelV
\end{align*}

The solid body movement implies that $\dive \fVelV = 0$. It is
furthermore easily shown that $\grad \fVelV$ is an antisymmetric
tensor, and therefore that $\grad \fVelV + \grad \fVelV^T=0$. Applying
both of these results to the shear stress tensor definition, we find
that it does not change in the relative frame:
\begin{align*}
  \shearStressT 
  &= \dynVisc \left(\grad \aVelV + \grad \aVelV^T\right) - \bulkVisc \left(\dive \aVelV\right) \identityT \\
  &= \dynVisc \left(\grad \rVelV + \grad \rVelV^T\right) - \bulkVisc \left(\dive \rVelV\right) \identityT
\end{align*}
% It is left as an exercise to show that
% \begin{align*}
%   \mathbf a \cdot \grad \fVelV = \rotV \times \mathbf a
% \end{align*}

The time derivatives relative to the frame of reference are different
to that in the absolute frame since a point attached to the frame
moves with respect to the absolute frame: as shown before the point
$P$ at $\xyzV$ in the absolute frame at time $t$, will have moved to
$P^\prime$ at $\xyzV^\prime = \xyzV + \fVelV dt$ at time $t +
dt$. Using Taylor expansion, we compute the relative time derivative
as
\begin{align*}
  \ddtR{a} &= \lim_{dt\rightarrow 0} \frac{1}{dt} 
  \left(a\left(\xyzV+\fVelV~dt,t+dt\right) - a\left(\xyzV,t\right)\right) \\
  &= \lim_{dt\rightarrow 0} \frac{1}{dt} \left(a\left(\xyzV,t\right) + \fVelV dt \cdot \grad a + dt \ddtA{a} + \mathcal O(dt^2) - a\left(\xyzV,t\right)\right) \\
  &= \ddtA{a} + \fVelV \cdot \grad a
\end{align*}

The time derivative of a vector $\mathbf a$ relative to the frame is
in principle transformed in the same way - if we keep the same system
of coordinates. However, we want to express it in the system of axes
attached to the relative frame. Since the orientation of the
``relative'' system of axes changes with respect to the absolute, we
see a change of direction, even if $\mathbf a$ stays constant in the
absolute frame:
\begin{align*}
  \ddtR{\mathbf a} = 
  \ddtA{\mathbf a} + 
  \fVelV \cdot \grad \mathbf a - 
  \rotV \times \mathbf a
\end{align*}

\subsection{Conservation of mass}

The continuity equation has exactly the same form in the relative as
in the absolute frame:
\begin{align*}
  \ddtA{\dens} + \dive \dens \aVelV = 0 &\Rightarrow
  \ddtR{\dens} - 
  \fVelV \cdot \grad \dens + 
  \dive \dens \left(\rVelV + \fVelV\right) = 0 \\
  &\Rightarrow
  \ddtR{\dens} + 
  \dive \dens \rVelV = 0
\end{align*}

\subsection{Conservation of momentum}

First we replace the absolute time derivative in the momentum
equation,
\begin{align*}
  \ddtA{\dens \aVelV} + 
  \dive \dens \aVelV \aVelV +  
  \grad p = 
  \dens \gravV + \dive \shearStressT 
  &\Rightarrow
  \ddtR{\dens \aVelV} - 
  \fVelV \cdot \grad \dens \aVelV + 
  \rotV \times \dens \aVelV + 
  \dive \dens \aVelV \aVelV +  
  \grad p = 
  \dens \gravV + \dive \shearStressT \\
  &\Rightarrow
  \ddtR{\dens \aVelV} - 
  \dive \dens \fVelV \aVelV + 
  \rotV \times \dens \aVelV + 
  \dive \dens \aVelV \aVelV +  
  \grad p = 
  \dens \gravV + \dive \shearStressT
\end{align*}
to arrive at equation governing the \emph{absolute momentum} in the
\emph{relative frame}. 
\begin{equation}
  \ddtR{\dens \aVelV} + 
  \rotV \times \dens \aVelV + 
  \dive \dens \rVelV \aVelV + 
  \grad p = 
  \dens \gravV + \dive \shearStressT
  \label{eq:relativeAbsoluteMomentum}
\end{equation}
Then by replacing the absolute velocity by $\aVelV = \rVelV + \fVelV$:
\begin{align*}
  &\ddtR{\dens \rVelV} + 
  \ddtR{\dens \fVelV} + 
  \rotV \times \dens \rVelV + 
  \rotV \times \dens \fVelV + 
  \dive \dens \rVelV \rVelV + 
  \dive \dens \rVelV \fVelV + \ldots \\
  &\Rightarrow \ddtR{\dens \rVelV} + 
  \dens \ddtR{\fVelV} + 
  \fVelV \ddtR{\dens} + 
  \rotV \times \dens \rVelV + 
  \rotV \times \dens \fVelV + 
  \dive \dens \rVelV \rVelV + 
  \fVelV \left(\dive \dens \rVelV\right) + \rotV \times \dens \rVelV + \ldots \\
  &\Rightarrow \ddtR{\dens \rVelV} + 
  \dens \ddtR{\fVelV} + 
  \fVelV \left(\ddtR{\dens} + \dive \dens \rVelV\right) +
  2 \rotV \times \dens \rVelV +
  \rotV \times \dens \fVelV + 
  \dive \dens \rVelV \rVelV + \ldots 
\end{align*}
we arrive at the momentum equation in the relative frame
\begin{equation}
  \ddtR{\dens \rVelV} + \dive \dens \rVelV \rVelV + \grad p +
  \underbrace{\dens \ddtR{\fVelV}}_{\mathbf F_f} + 
  \underbrace{2 \rotV \times \dens \rVelV}_{\mathbf F_{co}} + 
  \underbrace{\rotV \times \dens \fVelV}_{\mathbf F_{ce}} + 
  = \dens \gravV + \dive \shearStressT  
  \label{eq:relativeMomentum}
\end{equation}
Note the addition of three fictious forces $\mathbf F_f$, $\mathbf
F_{co}$ and $\mathbf F_{ce}$, compensating the (virtual) acceleration
\emph{seen} by the frame due to its motion. There are two cases of
interest. 
\begin{itemize}
\item For a frame moving at constant linear speed, the momentum
  equation retains the same form as in the absolute frame:
  \begin{equation}
    \ddtR{\dens \rVelV} + \dive \dens \rVelV \rVelV + \grad p
    = \dens \gravV + \dive \shearStressT  
    \label{eq:relativeMomentumLinear}
  \end{equation}
  as all fictitious forces are zero. 
\item for a frame rotating about axis \unitV{a} centered in the
  origin, we find
  \begin{figure}
    \centering
    \includegraphics{principles/rotatingFrame.tikz}
    \caption{Definition of polar coordinate system for a frame
      rotating around axis \unitV{\omega}, with radial \unitV{r} and
      azimuthal \unitV{u} directions, and $(z,r,\phi)$ the polar
      coordinates of point $P$.}
    \label{fig:polarRotatingFrame}
  \end{figure}
  \begin{equation}
    \ddtR{\dens \rVelV}  + 
    \dive \dens \rVelV \rVelV +
    \underbrace{2 \rotV \times \dens \rVelV}_{\mathbf F_{co}} + 
    \underbrace{- \dens \rot^2 r \unitV{r}}_{\mathbf F_{ce}}  + \grad p = \dens \gravV + \dive \shearStressT  
    \label{eq:relativeMomentumRotating}
  \end{equation}
  adding the fictitious \emph{Coriolis} $\mathbf F_{co}$ and
  \emph{centrifugal} force $\mathbf F_{ce}$. Here we have used $r$ as
  the distance to the axis and $\unitV{r}$ the local radial direction,
  as defined in figure \ref{fig:polarRotatingFrame}
\end{itemize}

\subsection{Conservation of energy}

We can rewrite the stagnation energy and enthalpy in the absolute
frame as
\begin{align*}
  \tEner &= \ener + \frac{1}{2} \aVel^2 = 
  \ener + \frac{1}{2} \rVel^2 - \frac{1}{2} \fVel^2 + \fVelV \cdot \aVelV \\
  \tEnth &= \enth + \frac{1}{2} \rVel^2 - \frac{1}{2} \fVel^2 + \fVelV \cdot \aVelV 
\end{align*}
We define the following functions
\begin{equation}
  \begin{split}
    \roenergy &= \ener + \frac{1}{2} \rVel^2 - \frac{1}{2}\fVel^2  \\
    \rothalpy &= \enth + \frac{1}{2} \rVel^2 - \frac{1}{2}\fVel^2  
  \end{split}
\end{equation}
which will be shown to generalize stagnation energy and enthalpy in a
moving frame of reference. First, we transform the different terms in
the energy conservation equation
\ref{eq:energyConservationDifferential} separately:
\begin{itemize}
\item the time derivative:
  \begin{align*}
    \ddtA{\dens \tEner} = 
    \ddtA{\dens \roenergy} + 
    \ddtA{} \dens \fVelV \cdot \aVelV  =
    \ddtR{\dens \roenergy} - 
    \underline{\fVelV \grad \dens \roenergy} + 
    \underline{\fVelV \cdot \ddtA{\dens \aVelV}} + 
    \dens \aVelV \ddtA{\fVelV} 
  \end{align*}
\item the enthalpy transport term:
  \begin{align*}
    \dive \dens \tEnth \aVelV 
    = 
    \dive \left(\dens \rothalpy + \dens\fVelV \cdot \aVelV\right) \aVelV 
    = 
    \dive \dens \rothalpy \rVelV +
    \underline{\fVelV \grad \dens \rothalpy} + 
    \underline{\fVelV \cdot \dive \dens \aVelV \aVelV} + 
    \dens \aVelV \aVelV : \grad \fVelV
  \end{align*}
  where the last term is zero since it is the internal product between
  a symmetric and an asymmetric tensor $\dens \aVelV \aVelV$
  resp. $\grad \fVelV$
\item Finally the shear stress work term transforms to 
  \begin{align*}
    \dive \shearStressT \cdot \aVelV &= 
    \dive \shearStressT \cdot \rVelV
    + \underline{\fVelV \cdot \dive \shearStressT}
  \end{align*}
\end{itemize}
Since since $\dens \rothalpy - \dens \roenergy = \pres$ the underlined terms
sum furthermore up to
\begin{align*}
  \fVelV \cdot \left(\ddtA{\dens \aVelV} + 
    \dive \dens \aVelV \aVelV
    + \grad p - \dive \shearStressT\right) = 0
\end{align*}
This leaves us with the equation 
\begin{equation}
  \ddtR{\dens \roenergy}  + 
  \dive \dens \rothalpy \rVelV + 
  \dens \aVelV \ddtA{\fVelV} 
  = 
  \dive \shearStressT \cdot \rVelV + 
  \dive \heatFluxV
\end{equation}
in which we find the roles of total internal energy \tEner~and
enthalpy \tEnth~are replaced by \roenergy~and \rothalpy; we
furthermore notice the work term resulting from the acceleration of
the frame. From now on, we will continue in frames with a steady
motion.

\subsection{Energy transfer, irreversibility and entropy}

Projecting the momentum equation again on the relative velocity
$\rVelV$, we find first that
\begin{align*}
  \rVelV \cdot \rotV \times \rotV \times \xyzV 
  &= - \rVelV \cdot \rot^2 r \unitV{r} = - \rVel_r \rot^2 r = - \rot^2 r \ddtM{r} = \ddtM{\eRot}
\end{align*}
we find
\begin{align*}
   \ddtM{\eKin}  +  \ddtM{\eRot} +  \ddtM{\eGrav} &= 
  \rVelV \cdot \frac{\grad p}{\dens} + 
  \rVelV \cdot \frac{\dive \shearStressT}{\dens} \\
  &= dW_{rev} + dW_{irr} 
\end{align*}
and therefore the mechanical energy is extended in much the same way
as the internal energy and enthalpy. The centrifugal force introduces a
potential, similar to gravity, which decreases with the distance to
the rotation axis:
\begin{align*}
  \eMech = \frac{v^2}{2} - \frac{u^2}{2} + \grav z = \eKin + \eRot + \eGrav
\end{align*}
The equation for the entropy has exactly the same form in the relative
frame:
\begin{equation}
  \ddt{\dens \entr} + \dive \dens \aVelV \entr = 
  \frac{\shearStressT : \grad \rVelV}{\temp} + \frac{\dive \heatFluxV}{\temp} 
  %=
  %\frac{\shearStressT : \grad \aVelV}{\temp} + \frac{\dive \heatFluxV}{\temp}
\end{equation}
and since $\shearStressT : \grad \rVelV = \shearStressT : \grad
\aVelV$\footnote{\shearStressT is symmetric, while $\grad \fVelV$ is
  antisymmetric.}, we find as expected that entropy does not depend on
the frame.

%%% Local Variables: 
%%% mode: latex
%%% TeX-master: t
%%% End: 
